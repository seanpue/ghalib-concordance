%!TEX TS-program = xelatex
%!TEX encoding = UTF-8 Unicode

\documentclass[12pt]{article}
\usepackage{geometry}                % See geometry.pdf to learn the layout options. There are lots.
\geometry{letterpaper}                   % ... or a4paper or a5paper or ... 
%\geometry{landscape}                % Activate for for rotated page geometry
%\usepackage[parfill]{parskip}    % Activate to begin paragraphs with an empty line rather than an indent
%\usepackage{graphicx}
%\usepackage{amssymb}
\usepackage{hyperref}
\usepackage{fontspec,bidi,bidipoem}
\newfontfamily\ur[Script=Arabic,Scale=1.4]{Jameel Noori Nastaleeq}
\setromanfont[Mapping=tex-text]{Hoefler Text}
\setsansfont[Scale=MatchLowercase,Mapping=tex-text]{Gill Sans}
\setmonofont[Scale=MatchLowercase]{Andale Mono}

%\title{Brief Article}
%\author{The Author}
%\date{}                                           % Activate to display a given date or no date

\begin{document}
\setRTL

\begin{table}[h]
\begin{tabular}{ll}
{\ur سلام آپ کیسے ہیں؟} & \href{http://www.texample.net/tikz/resources/}{url} \\
{\ur سلام آپ کیسے ہیں؟} & abc \\
{\ur سلام آپ کیسے ہیں؟} & abc 

\end{tabular}
\end{table}



hello
%\maketitle

% For many users, the previous commands will be enough.
% If you want to directly input Unicode, add an Input Menu or Keyboard to the menu bar 
% using the International Panel in System Preferences.
% Unicode must be typeset using a font containing the appropriate characters.
% Remove the comment signs below for examples.

% \newfontfamily{\A}{Geeza Pro}
% \newfontfamily{\H}[Scale=0.9]{Lucida Grande}
% \newfontfamily{\J}[Scale=0.85]{Osaka}

% Here are some multilingual Unicode fonts: this is Arabic text: {\A السلام عليكم}, this is Hebrew: {\H שלום}, 
% and here's some Japanese: {\J 今日は}.



\end{document}  